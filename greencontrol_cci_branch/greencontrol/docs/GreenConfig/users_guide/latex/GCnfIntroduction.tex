% (c) GreenSocs Ltd
% author: Christian Schroeder <schroeder@eis.cs.tu-bs.de>

\cleardoublepage
\chapter{Introduction}

The \GreenConfig project aims to provide a library that can be used to connect User defined configuration mechanisms to Tool configuration mechanisms, giving both flexibility. 

The diagram \ref{fig:config} shows an IP block connected, via the configuration library, to an ESL tool that is providing configuration from a database. 

The IP block has one API to the library. The library uses a different API to the ESL tool. 

\begin{figure}[htbp]
	\centerline{
		\includegraphics[width=10cm]{config.eps}} 
	\caption{IP block, ESL tool and GreenConfig Library}
	\label{fig:config}
\end{figure}

This allows IP designers to choose the configuration mechanism that best suits their style of IP creation, while staying independent of tools.  

\section{GreenControl}
See the \hypertarget{GCUsersGuide}{\href{http://www.greensocs.com/projects/GreenControl/docs/GCUsersGuide}{\GreenControl User's Guide}} and web page\footnote{\GreenControl project page:  \href{http://www.greensocs.com/projects/GreenControl}{http://www.greensocs.com/projects/GreenControl}} for a description of the \GreenControl framework which is the base of \GreenConfig.

\subsection{User APIs}

\paragraph{Config APIs}
Config APIs are \emph{special User APIs} whose task is the parameter configuration (\GreenConfig) and communicate with the \emph{ConfigPlugin}. 

\begin{itemize}
	\item Different API calls are translated into transactions with different command fields performing the actions inside the plugin.
	\item Config APIs receive transactions from the ConfigPlugin and process them, e.g. notification of value change events.
\end{itemize}

Config API implementations should be kept minimalistic. Each parameter access leads to a transaction to the ConfigPlugin. The other way around the developer of an API may want to cache parameter values locally.


\section{ConfigPlugin}
The ConfigPlugin provides all functionality required by the Config APIs. Hence it is an implementation of our requirement list (see appendix \ref{requirements}).


\section{Parameter management}
The configurable parameters are managed by the ConfigPlugin. To load and store parameters, it uses a Param-API which is connected to a storage implementation. This might be  

\begin{itemize}
	\item a simple map  
	\item an SQL database 
	\item ... 
\end{itemize}
